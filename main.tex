\documentclass[10pt,a4paper]{article}
 \usepackage[utf8]{inputenc}
 \usepackage[german]{babel}
 \usepackage[T1]{fontenc}
 \usepackage{amsmath}
 \usepackage{amsfonts}
 \usepackage{amssymb}
 \usepackage{hyperref}
 \usepackage{ bbold }
 \usepackage{MnSymbol}
\begin{document}
\tableofcontents
\section{Introduction}
This document is intended as an extension to the exercise generator:\\
 \url{https://github.com/TimJ2718/python-exercise-generator}\\
and as an introduction to linear algebra. 
 
\section{Determinant}
\subsection{Definition and conclusions}
The determinant function $\det(A)$ for a square matrix: $A \in K^{n \times n}$ is defined as the function which maps $K^{n \times n} \rightarrow K$ and satifies the following properties:\\
We write $A=(v_1,...v_n)$, where $v \in K ^{1 \times n}$.
\begin{itemize}
	\item $\det(v_1,...\lambda \cdot v_i,...v_n) = \lambda \det(v_1, ... v_i, ... v_n)$
	\item $\det(v_1,...v_i+w,...v_n) = \det(v_1, ...v_i,...v_n) + \det(v_1,...w,...v_n)$
	\item $v_i = v_j => \det(v_1,...v_i,...v_j,...v_n)=0$ 
	\item $\det(\mathbb{1})=1$ (Here $\mathbb{1}$ is the unity matrix)
\end{itemize}
Important conclusions are:
\begin{itemize}
	\item $\det(A)=0 $ <=> A does not have a full rank <=> The columns/rows are linear dependent
	\item $\det(A)=\det(A^T)$
	\item $\det(A \cdot B) = \det(A) \cdot \det(B)$
	\item $\det(v_1, ..., v_i, ..., v_j , ...v_n) = \det(v_1, ..., v_i+v_j, ...v_n)$
\end{itemize}
\subsection{Calculations}
\subsubsection{Explicit formulas}
$\det \left[ \left(\begin{array}{cc}
a_{11} & a_{12}\\
a_{21} & a_{22}
\end{array} \right) \right] = a_{11} a_{22} - a_{21} a_{12}$\\
$\det \left[ \left(\begin{array}{ccc}
a_{11} & a_{12} & a_{13}\\
a_{21} & a_{22} & a_{23}\\
a_{31} & a_{32} & a_{33}
\end{array} \right) \right] \\
\ \\
=a_{11} a_{22} a_{33} + a_{12}a_{23}a_{31}+a_{13}a_{21}a_{32}  -a_{31}a_{22}a_{13}-a_{32}a_{23}a_{11}-a_{33}a_{21}a_{12}$\\
\textbf{Examples}\\
$\det \left[ \left( \begin{array}{cc}
 2 & -3\\
 5 & 6
\end{array} \right) \right]
= 2 \cdot 6 - (5 \cdot (-3)) = 12 + 15 = 27$\\
$\det \left[ \left( \begin{array}{ccc}
 9 & 3 & 1\\
 2 & 6 & 3 \\
 6 & -1 & -1
\end{array} \right) \right]
= 9 \cdot 6 \cdot (-1) + 3 \cdot 3 \cdot 6 + 1 \cdot 2 \cdot (-1)
-6 \cdot 6 \cdot 1 - (-1) \cdot 3 \cdot 9 - (-1) \cdot 2 \cdot 3 
= -54 + 54 -2  - 36 +27 +6 = -5$\\




\subsubsection{Blockmatrix}
Let be $M \in K^{n \times n}$ and 0 the 0 matrix.\\
If A and C are quadratic matrices (C does not need to be quadratic) then the determinant of M is given by:
$M = \det\left[\left(\begin{array}{cc}
A & B \\
0 & C
\end{array}\right)\right]
= \det\left[A \right] \cdot \det \left[C \right]$\\
$M = \det\left[\left(\begin{array}{cc}
A & 0 \\
C & C
\end{array}\right)\right]
= \det\left[A \right] \cdot \det \left[C \right]$\\
\textbf{Example}\\
$\det \left[ \left( \begin{array}{ccccc}
2 & 3 & 4 & 8 & 2\\
6 & 2 & 3 & 9 & 4\\
0 & 0 & 9 & 3 & 1\\
0 & 0 & 2 & 6 & 3\\
0 & 0 & 6 &-1 &-1
\end{array}  \right) \right]
= \det \left[ \left( \begin{array}{cc}
2 & 3\\
6 & 2
\end{array}   \right) \right] \cdot
\det \left[ \left( \begin{array}{ccc}
9 & 3 & 1\\
2 & 6 & 3\\
6 &-1 &-1
\end{array}  \right) \right]
=-14 \cdot (-5) = 70 
$
\subsubsection{Triangular matrix}
$A = \left(\begin{array}{ccc}
a_{11} & \dots & a_{1n} \\
\vdots & \ddots & \vdots \\
a_{n1} & \dots & a_{nn}
\end{array}  \right)$\\
There are two kinds of triangular matrices:\\
A is an upper triangular matrix if all elements under the diagonal are zero: $a_{ij} =0 $ for $i > j$.\\
A is a lower trinangular matrix if all elements over the diagonal are zeros: $a_{ij}=0 $ for $i <j$.\\
The determinant of a triangular matrix is given by the product of the diagonal elements:
$\det(A)=\prod\limits_{i=1}^n a_{ii}$\\
\textbf{Example}\\
$\det \left[ \left( \begin{array}{ccc}
 2 & 5 & 4\\
 0 & 3 & 8\\
 0 & 0 & 4
\end{array} \right) \right] = 2 \cdot 3 \cdot 4 = 24; \;\;
\det \left[ \left( \begin{array}{ccc}
 2 & 0 & 0\\
 8 & 3 & 0\\
 5 & 4 & 4
\end{array} \right) \right] = 2 \cdot 3 \cdot 4 = 24$

\subsubsection{Laplace expansion}
$A = \left(\begin{array}{ccc}
a_{11} & \dots & a_{1n} \\
\vdots & \ddots & \vdots \\
a_{n1} & \dots & a_{nn}
\end{array}  \right)$\\
Define: $A_{ij}$ as the matrix where row i and the column j is removed:\\
$\tilde{A}_{ij}= \left( \begin{array}{cccccc}
a_{11} & \dots & a_{1,j-1} & a_{1,j+1} & \dots & a_{1n} \\
\vdots & \ddots & \vdots & \vdots & \udots &  \vdots \\
a_{i-1,1} & \dots & a_{i-1,j-1} & a_{i-1,j+1} & \dots & a_{i-1,n}\\
a_{i+1,1} & \dots & a_{i+1,j-1} & a_{i+1,j+1} & \dots & a_{i+1,n}\\
\vdots & \udots & \vdots & \vdots & \ddots &  \vdots \\
a_{n1} & \dots & a_{n,j-1} & a_{n,j+1} & \dots & a_{nn}
\end{array} \right)$\\
\ \\
The determinat of A is given by:\\
$\det(A) = \sum\limits_{j=1}^n (-1)^{i+j} a_{ij} \cdot \tilde{A}_{ij}$\\
$\det(A) = \sum\limits_{i=1}^n (-1)^{i+j} a_{ij} \cdot \tilde{A}_{ij}$\\
\textbf{Examples}\\
$\det \left[ \left( \begin{array}{ccc}
 2 & 0 & 2\\
 8 & 3 & 1\\
 5 & 4 & 4
\end{array} \right) \right] 
=(-1)^{1+1} \cdot 2  \cdot \det \left[ \left(\begin{array}{cc}
3 & 1\\
4 & 4
\end{array} \right) \right]
+ (-1)^{1+2} \cdot 0  \cdot \det \left[ \left(\begin{array}{cc}
8 & 1\\
5 & 4
\end{array} \right) \right]
+ (-1)^{1+3} \cdot 2  \cdot \det \left[ \left(\begin{array}{cc}
8 & 3\\
5 & 4
\end{array} \right) \right]
$\\
\ \\
\ \\
$\det \left[ \left( \begin{array}{ccc}
 2 & 0 & 2\\
 8 & 3 & 1\\
 5 & 4 & 4
\end{array} \right) \right] 
=(-1)^{1+2} \cdot 0  \cdot \det \left[ \left(\begin{array}{cc}
8 & 1\\
5 & 4
\end{array} \right) \right]
+ (-1)^{2+2} \cdot 3  \cdot \det \left[ \left(\begin{array}{cc}
2 & 2\\
5 & 4
\end{array} \right) \right]
+ (-1)^{3+2} \cdot 4  \cdot \det \left[ \left(\begin{array}{cc}
2 & 2\\
8 & 1
\end{array} \right) \right]
$

\section{Eigenvalue problem}
\subsection{Introduction}
A vector $v$ is called eigenvector of the linear map $\varphi$ if $\varphi(v)=\lambda \cdot v$.
Here the scalar $\lambda$ is called eigenvalue.\\
For obvious reasons $\varphi$ must be an endomorphismus and can be expressed as a quadratic matrix.
\subsection{Characteristical polynom}
Let be: $A \in K^{n \times n}$ and $v \in K^{1 \times n}$ and $\lambda \in K$.\\
$A \cdot V = \lambda V \rightarrow (\lambda \cdot \mathbb{1}-A) v =0$ (Here $\mathbb{1}$ is again the unity matrix.)\\
This can only be true if $(\lambda \cdot \mathbb{1} -A)$ has no full rank <=> $\det(\lambda \cdot \mathbb{1} - A) = 0$\\
We define $\chi(\lambda) = \det(\lambda \cdot \mathbb{1} - A)$ as the characterisitcal polynom.

\subsection{Eigenvalues}
The eigenvalues are given by the roots of the characteristical polynom. The multiplicity of the root is called algebraic multiplicity.
\subsection{Eigenvectors}
The eigenvectors to the eigenvalue $\lambda$ can be found by solving the following equation:\\ 
$(\lambda_i  \cdot \mathbb{1} - A) v = 0$
Every eigenvalue has at least one eigenvector but at most as many as the algebraic multiplicity. The number of linear indepenedent eigenvalues of a eigenvalue is called geometric multiplicity.
\subsection{Example}
Determine the characteristical polynom, the eigenvalues and the eigenvectors:\\
$\left(\begin{array}{cc}
 1 & 3\\
 0 & 4
\end{array}  \right)$\\
\ \\
\textbf{Chracteristical polynom:}\\
$\chi(\lambda)= \det\left(\begin{array}{cc}
\lambda -1 & -3 \\
0 & \lambda - 4
\end{array} \right)
= (\lambda-1) \cdot (\lambda -4) -0 \cdot (-3)=\lambda^2-5 \lambda +4$\\
\ \\
\textbf{Eigenvalues:}\\
$\chi(\lambda) = 0 <=> (\lambda-1) \cdot (\lambda-4) =0$\\
=> $\lambda_1 = 1$ : Algebraic multiplicity: 1\\
=> $\lambda_2 = 4$ : Algebraic multiplicity: 1\\
\ \\
\textbf{Hint}: You can check your result with the following relationship:\\ 
The sum of the eigenvalues is equal to the trace of the matrix.\\
$1+4 \stackrel{!}{=} \lambda_1 + \lambda_2$\\
(Eigenvalues that occur more than once are added multiple times according to their occurance)\\
(The trace of a matrix is given by the sum of the diagonal elements)\\
\ \\
\textbf{Eigenvectors:}\\
$\lambda_1=1:$\\
$\left(\begin{array}{cc}
0 & -3 \\
0 & -3 \\
\end{array} \right) \rightarrow
\left(\begin{array}{cc}
0 & 1 \\
0 & 0
\end{array} \right)
\rightarrow
v_1 = \left( \begin{array}{c}
1 \\ 0
\end{array} \right)
$\\
$\lambda_2=4:$\\
$\left(\begin{array}{cc}
3 & -3 \\
0 & 0 \\
\end{array} \right) \rightarrow
\left( \begin{array}{cc}
1 & - 1 \\
0 & 0
\end{array} \right) \rightarrow
v_2 = \left( \begin{array}{c}
1 \\1
\end{array} \right)
$ \\
\textbf{Hint}: You can plug in $v$ and $\lambda$ in the definition $A v = \lambda v$ and check if your calculation is right.
\subsection{Diagonalizability}
\end{document} 
